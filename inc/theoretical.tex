\section{Краткие теоретические сведения}

\subsection{Дискретные распределения}
\subsubsection{Общие теоретические сведения}

Полученную выборку ${x_1 , x_2 , x_3 , ..., x_N}$ упорядочить по возрастанию, определить частоты $n_i$ и относительные частоты (частости) $w_i$ , построить статистический ряд:\\


\begin{tabular}{|c|c|c|c|}
  \hline
  $x_i$ & $n_i$ & $w_i$ & $s_i$ \\
  \hline
  $x_{1}^{*}$ & $n_1$ & $w_1$ & $s_1$ \\
  \hline
  $x_{2}^{*}$ & $n_2$ & $w_2$ & $s_2$ \\
  \hline
  $\dots$ & $\dots$ & $\dots$ & $\dots$ \\
  \hline
  $x_{m}^{*}$ & $n_m$ & $w_m$ & $s_m$ \\
  \hline
  - & $\sum\limits_{i=1}^{m}n_i$ & $\sum\limits_{i=1}^{m}w_i$ & $-$ \\
  \hline
\end{tabular}\\
$x_i^* < x_j^*$ при $i < j$, $s_k = \sum\limits_{j=1}^k w_j$, $s_1 = w_1$, $s_m = 1$.

Полигон относительных частот – ломаная линия, соединяющая последовательно 
точки с координатами $(x_1^* , w_1 ), (x_2^* , w_2 ), \dots , (x_m^*
, w_m ).$

\textbf{Эмпирическая функция распределения}


\begin{equation*}
  F_N^{\text{Э}}(x)= \sum\limits_{x_i^*}w_i = 
\begin{cases}
0, &x<x_1^*\\
w_1, &x_1^* \leq x < x_2^* \\
w_1+w_2, & x_2^* \leq x < x_3^* \\
w_1+w_2+w_3, & x_3^* \leq x < x_4^* \\
\dots \\
1, & x \geq x_m^*
\end{cases}
\end{equation*}

\newpage
\textbf{Выборочное среднее}

$\overline{x} = \sum\limits_{i=1}^{m} x_i^* w_i$

\textbf{Выборочная дисперсия}

$D_B = \sum\limits_{i=1}^m (x_i^*- \overline{x})^2 w_i$


\textbf{Выборочный момент k-ого порядка}

$\overline{\mu_k} = \sum\limits_{i=1}^m (x_i^*)^k w_i$

\textbf{Выборочный центральный момент k-ого порядка}

$\overline{\mu_k^o} = \sum\limits_{i=1}^m (x_i^*- \overline{x})^k w_i$

\textbf{Выборочное среднее кврадратическое отклонение}

$\overline{\sigma} = \sqrt{D_B}$

\textbf{Выборочная медиана}
\begin{equation*}
\overline{M_e} = 
\begin{cases}
	x_i^*, & F_N^{\text{Э}} (x_{i-1}^*) < 0,5 <  F_N^{\text{Э}} (x_{i}^*)\\
	\cfrac{1}{2} (x_i^{*} + x_{i+1}^*), &F_N^{\text{Э}} (x_{i}^*) = 0,5
\end{cases}
\end{equation*}

\textbf{Выборочная мода} - то значение $x_i$, которому соответствует максимальная частота.\\
$\begin{cases}
	\text{если } n_i = \max n_k >n_j, i \not= j, & \Rightarrow \overline{M_0}=\{ x_i^* | n_i = \max n_k \} \\
	\text{если } n_i = n_i + 1 = \dots = n_{i+j} = \max n_k, & \Rightarrow \overline{M_0} = \cfrac{1}{2} (x_i^* + x_{i+1}^*) \\
	\text{если } n_i = \max n_k > n_l, i<l<j, & \Rightarrow
	 \overline{M_0} \text{ -не существует}
\end{cases}$

\textbf{Выборочный коэффициент асимметрии}

$\overline{a_s} = \cfrac{\overline{\mu_3^o}}{\overline{\sigma}^3}$

\textbf{Выборочный коэффициент эксцесса}

$\overline{\varepsilon_k} = \cfrac{\overline{\mu_4^o} }{\overline{\sigma}^4} - 3$

\textbf{Ряд распределения} - структурная группировка с целью выделения характерных свойств и закономерностей изучаемой совокупности.

\textbf{Математическое ожидание} – понятие среднего значения случайной величины в теории вероятностей.

\textbf{Дисперсия} – отклонение величины от ее математического ожидания.

\textbf{Среднеквадратическое отклонение} – показатель рассеивания значений случайной величины отноительно ее математического ожидания.

\textbf{Мода} – значение во множестве наблюдений, которое встречается наиболее часто.

\textbf{Медиана} – возможное значение признака, которое делит вариационный ряд выборки на две равные части.

\textbf{Коэффициент ассиметрии} используется для проверки распределения на симметричность, а также для грубой предварительной проверки на нормальность.

Если плотность распределения симметрична, то выборочный коэффициент ассиметрии равен нулю, если левый хост распределения тяжелее – больше нуля, легче – меньше.


\textbf{Коэффициент эксцесса} используется для проверки на нормальность.

Нормальное распределение имеет нулевой эксцесс. Если хвосты распределения «легче», а пик острее, чем у нормального распределения, то коэффициент эксцесса положительный; если хвосты распределения «тяжелее», пик «приплюснутый», чем у нормального распределения, то отрицательный.


\subsubsection{Биномиальное распределение}

\textbf{Биномиальное распределение}- распределение количества «успехов» в последовательности из n независимых случайных экспериментов, таких что вероятность «успеха» в каждом из них равна p.\\
Ряд распределения: $p_m = C_n^m p^m q^{n−m}$ \\
Математическое ожидание: $np$ \\
Дисперсия: $npq$ \\
Среднеквадратическое отклонение: $\sqrt{npq}$ \\
Мода: $\big[(n+1)p\big]$, если $(n+1)p$ – дробное; 
  $(n+1)p-\frac{1}{2}$, если $(n+1)p$ – целое \\
Медиана:  Round(np) \\ %$np$ из $\{[np]-1, [np],[np]+1\}$ \\
Коэффициент ассиметрии: $\cfrac{q-p}{\sqrt{npq}}$\\
Коэффициент эксцесса: $\cfrac{1- 6pq}{npq} $\\

\subsubsection{Геометрическое распределение}

\textbf{Геометрическое распределение} – распределение величины, равной количеству испытаний случайного эксперимента до наблюдения первого «успеха». \\
Ряд распределения: $P = q^n p$\\
Математическое ожидание: $\cfrac{q}{p}$\\
Дисперсия: $\cfrac{q}{p^2}$\\
Среднее квадратичное отклонение: $\sqrt{\cfrac{q}{p^2}}$\\
Мода: $0$ \\
Медиана: $\Big[-\cfrac{\ln2}{\ln q}-1\Big]+1$\\
Коэффициент ассиметрии: $\cfrac{2-p}{\sqrt{1-p}}$\\
Коэффициент эксцесса: $6+ \cfrac{p^2}{1-p}$\\


\subsubsection{Распределение Пуассона}

\textbf{Распределение Пуассона} - вероятностное распределение дискретного типа, моделирует случайную величину, представляющую собой число событий, произошедших за фиксированное время, при условии, 
что данные события происходят с некоторой фиксированной средней интенсивностью и независимо друг от друга. \\
Ряд распределения: $P= \cfrac{\lambda^k}{k!} e^{-\lambda}$\\
Математическое ожидание: $\lambda$\\
Дисперсия: $\lambda$\\
Среднеквадратическое отклонение: $\sqrt{\lambda}$\\
Мода: $[\lambda]$\\
Медиана: $\Big[ \lambda + \cfrac{1}{3} - \cfrac{0.002}{\lambda} \Big]$\\
Коэффициент ассиметрии: $\lambda^{-\frac{1}{2}}$\\
Коэффициент эксцесса: $\lambda^{-1}$\\

\subsubsection{Гипергеометрическое распределение}

\textbf{Гипергеометрическое распределение}\\ 
Ряд распределения: $p_k = \frac{C_{K}^k \cdot C_{M-K}^{m-k}}{C_{M}^m}, k=0,1, \dots, m$\\
Математическое ожидание: $\frac{mK}{M}$\\
Дисперсия: $ \frac{mK(M-K)(M-m)}{(M-1)M^2} $\\
Среднеквадратическое отклонение: $ \frac{1}{M} \sqrt{ \frac{mK(M-K)(M-m)}{(M-1)} } $\\
Мода: $[ \frac{(K+1)(m+1)}{M+2} ]$\\
Медиана: $\begin{aligned}
  l \text{ при } & \sum\limits_{k=0}^{l-1} < 0,5 < \sum\limits_{k=0}^{l};\\
  l + 0,5 \text{ при } & \sum\limits_{k=0}^{l}=0,5
\end{aligned}$\\
Коэффициент ассиметрии: $ \frac{ (M-2K)(M-2m) }{M-2} \cdot \sqrt{ \frac{M-1}{mK(M-K)(M-m)} } $\\
Коэффициент эксцесса: $ \big[ \frac{(M-1)M^2}{ m(M-2)(M-3)(M-m) } \big] \times 
\big[ \frac{ M(M+1)-6M(M-m) }{K(M-K)} + \frac{3m(M+6)(M-m)}{M^2} - 6 \big]$\\

\subsubsection{Равномерное распределение на множестве $\{0,\dots ,n-1\}$}

\textbf{Равномерное распределение на множестве $\{0,\dots ,n-1\}$}\\ 
Ряд распределения: $p_k = \frac{1}{n}, k =0, \dots , (n-1)$\\
Математическое ожидание:  $\frac{n-1}{2}$\\
Дисперсия: $\frac{n^2 - 1}{12}$\\
Среднеквадратическое отклонение: $\frac{1}{2} \sqrt{ \frac{n^2-1}{3} }$\\
Мода: $\frac{n-1}{2}$\\
Медиана: $\frac{n-1}{2}$\\
Коэффициент ассиметрии: $0$\\ 
Коэффициент эксцесса: $ - \frac{ 6(n^2+1) }{ 5(n^2 -1) } $\\


\subsubsection{Средства языка программирования}%

\textbf{Функции генерации распределений}

Функция для генерации биномиального распределения.
\begin{lstlisting}[language=Python]
from scipy.stats import binom
r = binom.rvs(n, p, size=200)
\end{lstlisting}
Принимает на вход параметры n,p и размер выборки. Возвращает массив сгенерированных значений.


Функция для генерации геометрического распределения.
\begin{lstlisting}[language=Python]
from scipy.stats import geom 
r = geom.rvs(p, size=200)
\end{lstlisting}
Принимает на вход параметры p и размер выборки. Возвращает массив сгенерированных значений.
Функция генерирует первое распределение. 

Функция для генерации распределения Пуассона.
\begin{lstlisting}[language=Python]
from scipy.stats import poisson 
r = poisson.rvs(lambda, size=200)
\end{lstlisting}
Принимает на вход параметры lambda и размер выборки. Возвращает массив сгенерированных значений.

Функция для генерации равномерного распределения.
\begin{lstlisting}[language=Python]
from scipy.stats import  randint
r = randint.rvs(0, n, size=200)
\end{lstlisting}
Принимает на вход два параметра интервала и размер выборки. Возвращает массив сгенерированных значений.

Функция для генерации чисел по гипергеометрическому закону.
\begin{lstlisting}[language=Python]
from scipy.stats import hypergeom 
r = hypergeom.rvs(M, K, m, size=200)
\end{lstlisting}
Принимает на вход параметры M, K, m и размер выборки. Возвращает массив сгенерированных значений.

\subsection{Непрерывные распределения}
\subsubsection{Общие теоретические сведения}

При построении группированной выборки (интервального вариационного ряда) число интервалов $[a_0, a_1], (a_0, a_1], \dots, (a_{m-1}]$
определяется по формуле Стерджеса $m=1+[log_{2}N], a_0=x_{(1)}, a_m = x_{(N)}, a_k - a_{k-1} = d/m,\ k=1,\dots m$, где
$d=a_m - a_0$. Для показательного распределения положить $a_0=0$, для равномерного распределения на отрезке $[a,b]$ положить 
$a_0=a, a_m = b$.

\textbf{Выборочное среднее}: $\overline{x} = \frac{1}{N} \sum\limits_{i=1}^m x_{i}^* \cdot n_i = \sum\limits_{i=1}^m x_{i}^* \cdot w_i$

\textbf{Выборочная дисперсия с поправкой Шеппарда}: $s_{B}^2 = \sum\limits_{i=1}^m (x_{i}^* - \overline{x})^2 \cdot w_i - \frac{h^2}{12}$,
где $h=(a_m - a_0)/m$

\textbf{Выборочное среднее квадратичное отклонение}: $\sigma = \sqrt(s_{B}^2)$

\textbf{Выборочная мода (для одного модального интервала)}: \\
$\overline{M}_0 = a_{k-1} + h \frac{w_k - w_{k-1}}{2w_k - w_{k-1} - w_{k+1}}$\\
$a_{k-1}$- левая граница модельного интервала
$w_k$ - относительная частота на модальном интервале
$w_{k-1}, w_{k+1}$ - относительные частоты интервалов слева и справа от модального интервала

\textbf{Выборочная медиана}:\\
$\overline{M_e} = a_{k-1} + \frac{h}{w_k} (\frac{1}{2} - \sum\limits_{i=1}^{k-1} w_i)$, если $\sum\limits_{i=1}^{k-1}w_i < \frac{1}{2} < \sum\limits_{i=1}^{k}w_i$;\\
$\overline{M_e} = a_k$, если $\sum\limits_{i=1}^k w_i =\frac{1}{2}$

\textbf{Выборочный момент $k$-ого порядка}: $\overline{\mu} = \overline{x^k} = \sum\limits_{i=1}^m (x_{i}^*)^k \cdot w_i \cdot \overline{\mu_1} = \overline{x}$

\textbf{Выборочный центральный момент $k$-ого порядка}: $\overline{\mu}_{k}^0 = \sum\limits_{i=1}^m (x_{i}^* - \overline{x})^k \cdot w_i, \overline{\mu}_{1}^0 = 0, \overline{\mu}_{2}^0 = D_B = \overline{\mu}_{2} - (\overline{\mu}_1)^2$

\textbf{Выборочный коэффициент ассимметрии}: $\overline{y}_1 = \frac{\overline{\mu}_{3}^0}{\overline{\sigma}^3}$

\textbf{Выборочный коэффициент эксцесса}: $\overline{y}_2 = \frac{\overline{\mu}_{4}^0}{\overline{\sigma}^4} -3$ 


\subsubsection{Нормальное распределение}
\textbf{Нормальное распределение} Плотность распределения: $f(x) = \frac{1}{\sigma \sqrt{2\pi}} e^{-\frac{(x-a)^2}{2\sigma^2}}$
Функция распределения: $F(x) = \frac{1}{\sqrt{2\pi}} \int\limits_{-\infty}^x e^{-\frac{(t-a)^2}{2\sigma^2}} dt$
Математическое ожидание: $a$
Дисперсия: $\sigma^2$
Среднеквадратическое отклонение: $\sigma$
Мода: $a$
Медиана: $a$
Коэффициент ассиметрии: $0$
Коэффициент эксцесса: $0$

\subsubsection{Показательное распределение}
\textbf{Показательное распределение} 
Плотность распределения: $\lambda e^{-\lambda x}$
Функция распределения: $1 - e^{-\lambda x} $
Математическое ожидание:  $\lambda^{-1}$
Дисперсия:  $\lambda^{-2}$
Среднеквадратическое отклонение: $\lambda^{-1}$
Мода: $0$
Медиана: $\frac{ln2}{\lambda}$
Коэффициент ассиметрии: $2$
Коэффициент эксцесса: $6$ 

\subsubsection{Равномерное распределение на отрезке $[a,b]$}
\textbf{Равномерное распределение на отрезке $[a,b]$} 
Плотность распределения: $\frac{1}{b-a}, a\le x \le b$
Функция распределения: $\frac{x-a}{b-a}, a \le x < b$
Математическое ожидание: $\frac{a+b}{2}$
Дисперсия: $\frac{(b-a)^2}{12}$
Среднеквадратическое отклонение: $\frac{b-a}{2\sqrt{3}}$
Мода: $\frac{a+b}{2}$
Медиана: $\frac{a+b}{2}$
Коэффициент ассиметрии: $0$
Коэффициент эксцесса: $-\frac{6}{5}$
