\section{Задание}

\textbf{Задание 1.} Получить выборку, сгенерировав 200 псевдослучайных чисел, 
распределенных по биномиальному закону с
параметрами $n$ и $p$:\\
$p_k=C_{n}^{k} \cdot p^k \cdot q^{n-k}, k=0,1, \dots, n.$  
$\qquad n = 5 + V mod 19, p =0,2 + 0,005V$

\textbf{Задание 2.} Получить выборку, сгенерировав 200 псевдослучайных чисел, 
распределенных по геометрическому закону с параметром $p$:\\ 
$p_k = q^k \cdot p, k = 0,1, \dots \qquad p = 0,2 + 0,005V$

\textbf{Задание 3.} Получить выборку, сгенерировав 200 псевдослучайных чисел, 
распределенных по закону Пуассона с параметром $\lambda$. 
$p_k = \frac{\lambda^{k}}{k!} e^{-\lambda}, k =0,1, \dots$ 
$\qquad \lambda = 0,5 + 0,01 \cdot V$

\textbf{Задание 4.} Получить выборку, сгенерировав 200 псевдослучайных чисел, 
распределенных равномерно на множестве $\{ 0, \dots , n-1 \}$: \\
$p_k = \frac{1}{n}, k=0, \dots , (n-1). \qquad n = b + V mod 28$

\textbf{Задание 5.} Получить выборку, сгенерировав 200 псевдослучайных чисел, 
распределенных по гипергеометрическому закону с параметрами $M, K, m$: \\ 
$p_k = \frac{C_{K}^k \cdot C_{M-K}^{m-k}}{C_{M}^m} , k = 0,1, \dots , m . \qquad$
$m = 5+V mod 7, \quad M = 2m+3 \quad K=m+1+Vmod 2$

Следуя указаниям для всех выборок в Заданиях 1-5:
Построить:
\begin{enumerate}[label=\arabic*)]
	\item статистический ряд;
	\item полигон относительных частот;
	\item график эмпирической функции распределения;
\end{enumerate}

Найти:
\begin{enumerate}[label=\arabic*)]
	\item выборочное среднее;
	\item выборочную дисперсию;
	\item выборочное среднее квадратическое отклонение;
	\item выборочную моду;
	\item выборочную медиану;
	\item выборочный коэффициент асимметрии;
	\item выборочный коэффициент эксцесса.
\end{enumerate}

Составить таблицы:
\begin{enumerate}[label=\arabic*)]
	\item сравнения относительных частот и теоретических вероятностей;
	\item сравнения рассчитанных характеристик с теоретическими значениями. 
\end{enumerate}


\textbf{Задание 6.} Получить выборку, сгенерировав 200 псевдослучайных чисел, распределенных по
нормальному закону с параметрами $a=(-1)^V \cdot 0,05 \cdot V$ и $\sigma^2$, где $\sigma = 0,005 \cdot V + 1$.

\textbf{Задание 7.} Получить выборку, сгенерировав 200 псевдослучайных чисел, распределенных по
показательному закону с параметром $\lambda = 2 + (-1)^V \cdot 0,01 \cdot V$

\textbf{Задание 8.} Получить выборку, сгенерировав 200 псевдослучайных чисел, распределенных
равномерно на отрезке $[a, b]$, где $a = (-1)^V \cdot 0,02 \cdot V, b = a + 6$

Следуя указаниям для всех выборок в Заданиях 6-8:
Построить:
\begin{enumerate}[label=\arabic*)]
	\item интервальный ряд;
  \item ассоциированный статистический ряд; 
	\item гистограмму относительных частот с наложенным на нее графиком плотности соответствующего 
    распределения; 
	\item график эмпирической функции распределения;
\end{enumerate}

Найти:
\begin{enumerate}[label=\arabic*)]
	\item выборочное среднее;
	\item выборочную дисперсию с поправкой Шеппарда;
	\item выборочное среднее квадратическое отклонение;
	\item выборочную моду;
	\item выборочную медиану;
	\item выборочный коэффициент асимметрии;
	\item выборочный коэффициент эксцесса.
\end{enumerate}

Составить таблицы:
\begin{enumerate}[label=\arabic*)]
	\item сравнения относительных частот и теоретических вероятностей;
	\item сравнения рассчитанных характеристик с теоретическими значениями. 
\end{enumerate}
